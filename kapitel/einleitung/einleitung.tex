\section{Introduction} \label{introduction}

The need for a shelter is one of the most fundamental human needs. herefore, housing sector is a major area of interest for both governments and people. Developments in housing market can affect both consumers and credit institutes. Considering its importance, the housing market plays a central role in monetary and fiscal policies of many countries. In order to better understand this sector of economy, one needs to be able to model it. Models are used to simulate the behavior of the real system for prediction and planning purposes. In this work, we firstly examine an existing well known model in this area with the real data from German economy. Secondly, identify a model from data to get a better understanding of the system.

\subsection{Problem Definition}
Understanding a system is a prerequisite to simulation or control of a system. Housing sector of economy is no exception to this rule. How can governments or central banks respond to  rising house prices? Or How housing prices are affected if the central bank decides to increase the interest rate. This questions can only be answered, if there exists a sound model which is close enough to the real system.


\subsection{Objectives}
In this paper we explain the relationship between housing price and monetary and fiscal policies in Germany .
The major objectives of the study are: Firstly, to explain the behavior of the housing market in relationship to monetary policies in Germany using the previously mentioned housing related monetary transmission channels. Secondly, to analyze the data and derive a model using computational macro economics methods. 

\subsection{Methodology}
Firstly the transmission channels proposed by Mishkin \footcite[See.][]{Mishkin2007} illustrated in Figure \ref{fig:HousingTransmission} are used to explain the macro economics and housing related data in Germany. Secondly, an \ac{SVAR} model is proposed and identified using the data evidence. Details about each of these two methods are explained in corresponding chapters.




