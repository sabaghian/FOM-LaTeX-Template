\section{Introduction} \label{introduction}

\ac{PM} is one of the key aspects of people and business management. The performance of the organization depends on the performance of every staff on every level working in that organization. Continuous improvements of people will enhance the performance of the organization. An effective \ac{PM} helps the employees perform better and align their contributions to the goals of the organization. \ac{PM} gives the supervisors a model to plan, analyze, and maintain the process of improving the performance of their staff \footcite[See. ][]{Cadwell2000}.  

\subsection{Problem Definition}
Every \ac{PM} process can fail or succeed. The key to a successful \ac{PM} is the implementation. In this work, we firstly define the process of \ac{PM} and its different aspects. Then, the correct implementation of \ac{PM} is discussed. Finally, a list of \enquote{to do} and \enquote{not to do} is given to summarize how can one improve the \ac{PM} inside an organization.


\subsection{Objectives}
The main objective of this work is to briefly introduce the process of \ac{PM} and a plan for how to correctly implement it.

\subsection{Methodology}
This work is mainly a literature review. However, the author's opinion from his personal experience as a line manager is given occasionally.





