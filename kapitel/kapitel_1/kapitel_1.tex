\newpage
\section{Performance Management Definition} \label{definition}
According to Armstrong \enquote{\ac{Performance management} can be defined as a systematic process for
improving organizational performance by developing the performance of
individuals and teams. It is a means of getting better results from the
organization, teams and individuals by understanding and managing
performance within an agreed framework of planned goals, standards
and competence requirements},\footcite[See][S. 1]{Armstrong2006}.
\ac{PM} is a partnership between employee and the supervisor. It is a mutual process which cannot be successfully implemented by a supervisor alone \footcite{Cadwell2000}.  
 The goal of the performance management is to establish a systematic approach to improve overall performance of the organization. This can be achieved only if the organization goals are well defined and are aligned with individual objectives. Performance management makes sure that individual have a clear idea  of what they are expected to do and receive accurate feedback and coaching enabling them to achieve those goals. \footcite[See][]{Armstrong2006}.

The main aspects of performance management are agreement, measurement, feedback, positive reinforcement
and dialogue\footcite[See][]{Armstrong2006}.

The focus should be mostly on the future and planning rather than criticizing past performance. It should be implemented mostly with continous dialogues instead of rating scales. \ac{PM} is concerned not only with output but also with the input and process of how the goals are achieved. This way it is possible to estimate what can be done to improve the results in further steps. Measurement is a key factor because it is difficult to manage sth which is not measurable. Performance Management concerns about not only managers but all the stake holders including employees, costumers, suppliers. That is why it is recommended that the employees opinion are involved in creation of objectives.\footcite[See][]{Armstrong2006}.
 	 	
\subsection{Performance Management Benefits} \label{benefits}
As we mentioned earlier \ac{PM} improves the performance of the employee in an organization. In my opinion, this is mainly due to employees improved motivation and not because his performance is monitored. As a developer, I have happily spent extra hours at work without any additional payment to finish a project that I thought made sense. A project that I was present in planning phase and my ideas were taken into account. Moreover, a correctly implemented \ac{PM} leads to improved communication in the organization. It helps to align individual's goals with that of the company. Therefore, the employee's self management would be improved and consequently his job satisfaction \footcite[See][]{Cadwell2000}. 

\subsection{Performance Management vs Performance Appraisal}
Performance Appraisal is usually the top down approach of rating the employees at some planned time. In Contrary, performance management is a continuous process and mostly focuses on the future instead of judging the past. Therefore, performance management cannot be solely owned by \ac{HR} department but also line managers are involved.
\subsection{Performance Management Process}
the process management process includes \footcite[See][]{Armstrong2006}. 
\begin{enumerate}
\item Planning
\item Acting
\item Monitoring
\item Reviewing
\end{enumerate}
As previously mentioned, planning in performance management is not in the form of a top down approach but instead it is in the form of agreement. Once the planned is agreed upon, actions need to be taken to account to achieve it. The achievement of the objectives must be constantly monitored and this progress needs to be assessed for improving further planning.
In figure \ref{fig:PerformanceManagementCycle} the cycle is explained.
\begin{figure}
\caption{Performance Management Cycle}
\label{fig:PerformanceManagementCycle}
\includegraphics[width=0.9\textwidth]{PerformanceManagementCycle}
\\
\cite[Source: See][]{Armstrong2006}
\end{figure}

\begin{figure}
\caption{Performance Management Sequence}
\label{fig:PerformanceManagementSequence}
\includegraphics[width=0.9\textwidth]{PerformanceManagementSequence}
\\
\cite[Source: See][]{Armstrong2006}
\end{figure}

In figure \ref{fig:PerformanceManagementSequence} the sequence is explained. In the role definition action, parties agree on key results and required competence to achieve that. The performance agreement action is about defining the objectives for individuals and methods to measure the objectives and the required competence. The performance improvement plan, explains the necessary action for individuals to be taken when their performance should improve. Performance review is a formal session to review the performance over a period of time and learn for the next revised performance agreement.

Performance matrices can be used for management appraisal. However, this should be distinguished with appraisal rating since the goal here is to help individuals find tasks that they are good at and help them improve in those tasks.\footcite[See][]{Armstrong2006}. 

\subsubsection{Performance and Development Planning}
Help people to achieve the objectives of the tasks they agreed on. In the first step the line manager and the team member make sure that they have a mutual understanding of the role. They agree on not only the key results bur also the way it is done including the required competencies and core values that needs to be uphold.\footcite[See][]{Armstrong2006}. 
 I can say from my personal experience that this step is not taken very seriously in many companies. Usually the manager has a vague idea of what need to be accomplished and the task is assigned to whoever is less occupied at this particular moment.  I find the emphasis on input very important. For example, one can consider a software company. One objective can be developing a simulation platform for a robot manipulator in 11 months. However, it is very important to remember that this product may need to be further developed in some points in the future. Therefore, the source code should be implemented with some certain coding and documentations standard. As a result, the developer should welcome other team member's opinions and even critics. If the individual is not fully aware of this expected competence, he may consider his teammates engagements as not productive or even annoying. 
\subsubsection{Continuous Monitoring}

As already mentioned in previous sections, performance management is a continuous process. Managers should not wait for an official review meeting for giving feedbacks or even receiving feedbacks from the individuals. Sometimes the objectives needs to be updated, or some obstacles prevents the individuals from accomplishing their tasks. The line manager identifies these problems as soon as possible and reacts on them. When giving feedback one needs to consider the following points\footcite[See][]{Chappelow2019}. : 
\begin{itemize}
\item Harsh Feedback does not help. Giving frequent negative feedback leads to defensive reaction.  For example, interrupting someone by telling that idea never works.
\item The critical role of positive feedback should not be ignored.
\item Telling people how to solve the problem is the wrong approach.  This is not the same as coaching.
\end{itemize}

\subsubsection{Performance Review}
 One or two formal review secessions per year are still inevitable. These sessions are used to identify main performance and development issues. Conducting a review session requires management skills otherwise the emergence of hostile attitude is very probable.\footcite[See][]{Armstrong2006}. I think the main focus of this meeting should be on future. What can we improve in the future is the question which needs to be answered at the end. Indeed, the review session and the way it is conducted can be reviewed and improved.


