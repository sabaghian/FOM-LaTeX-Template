\newpage
\section{Modern Performance Management} \label{PM2}

Some recent authors criticize the classical performance management methods. Colquitt ~\footcite[See. ][]{Colquitt2017} calls the classical performance management `PM 1` and he explains what should be stopped in `PM 1` and what needs to be started. He calls the new process `PM 2` which is a more updated version and more suitable for modern life.
\subsection{Things To Stop}

PM 1 requires organizations to monitor performance and measure results. However, experiments show that the judgments are usually affected with many factors and they are not accurate enough for such a process. Relative feedback is more damaging in team work than constructive. It can lead to demotivation or even sabotaging the teammates. Besides that, people are not always in full control of the situation and the uncontrollable factors can affect the performance measurements which has nothing to do with individual ~\footcite[See. ][]{Colquitt2017}. It has happened to me several times that I have been criticized in a review process for the results which has been totally out of my control and the manager took my explanation as an excuse.

\subsection{Things To Start}

It is important for the companies to understand that they cannot buy commitment. If they really want their employees to commit they need to share the common belief and have a clear understanding of the objectives and the path to achieve them. They need to feel involved in something bigger than their everyday work. However, companies need to be careful with the weight they put on objectives since it can lead the a mentality of achieving a goal no matter the price ~\footcite[See. ][]{Colquitt2017}. In my opinion, competition in a job which requires creativity does more harm than good. People stop sharing information instead of helping each other to grow for the purpose of winning the competition.





